%%%%%%%%%%%%%%%%%%%%%%%%%%%%%%%%%%%%%%%%%
% Medium Length Professional CV
% LaTeX Template
% Version 2.0 (8/5/13)
%
% This template has been downloaded from:
% http://www.LaTeXTemplates.com
%
% Original author:
% Trey Hunner (http://www.treyhunner.com/)
%
% Important note:
% This template requires the resume.cls file to be in the same directory as the
% .tex file. The resume.cls file provides the resume style used for structuring the
% document.
%
%%%%%%%%%%%%%%%%%%%%%%%%%%%%%%%%%%%%%%%%%

%----------------------------------------------------------------------------------------
%	PACKAGES AND OTHER DOCUMENT CONFIGURATIONS
%----------------------------------------------------------------------------------------

\documentclass{resume} % Use the custom resume.cls style

\usepackage[left=0.75in,top=0.6in,right=0.75in,bottom=0.6in]{geometry} % Document margins

\usepackage{enumitem}
\setlist{leftmargin=0em}
\renewcommand{\labelitemi}{$\cdot$}

%%MINE THAT I ADDED

\usepackage{lipsum}
\usepackage{datetime}
\usepackage{hyperref}
\usepackage{ragged2e}

\newcommand\blfootnote[1]{%
  \begingroup
  \renewcommand\thefootnote{}\footnote{#1}%
  \addtocounter{footnote}{-1}%
  \endgroup
}

%---------------------------

\name{Katherine A. Keith} % Your name
\address{
%(406)~$\cdot$~539~$\cdot$~3888 \\ 
kkeith@cs.umass.edu} % Your phone number and email

\begin{document}

%%----------------------------------------------------------------------------------------
%%	RESEARCH INTERESTS
%%----------------------------------------------------------------------------------------
%
%\begin{rSection}{Research Interests}
%Natural language processing, machine learning, computational social science 
%\end{rSection}

%----------------------------------------------------------------------------------------
%	EDUCATION SECTION
%----------------------------------------------------------------------------------------

\begin{rSection}{Education}

{\bf University of Massachusetts Amherst} \hfill {2016--Present} \\ 
M.S.~/~Ph.D. in Computer Science (in progress) \hfill {\em Amherst, MA} \\
GPA 3.88/4.0 \\

{\bf Lewis \& Clark College} \hfill {2011--2015} \\ 
B.A. in Mathematics with departmental honors, summa cum laude \hfill {\em Portland, OR} \\
Minor: Chinese \\
GPA 3.95/4.0 \\
\end{rSection}

%----------------------------------------------------------------------------------------
% RESEARCH EXPERIENCE SECTION
%----------------------------------------------------------------------------------------

\begin{rSection}{Research Experience}

\begin{rSubsection}{Graduate Research Assistant}{September 2016--Present}{University of Massachusetts, Amherst}{Amherst, MA}
\item Natural language processing, machine learning, and computational social science research
\item Advisor: Dr. Brendan O'Connor 
\end{rSubsection}

%------------------------------------------------

\begin{rSubsection}{Undergraduate Research Assistant}{May 2014--August 2014}{Lewis \& Clark College}{Portland, OR}
\item Developed an agent-based simulation of intergenerational wealth transfer in medieval England
\item Funding: Andrew W. Mellon Collaborative Student-Faculty Research Grant
\item Advisor: Dr. Clifford Bekar
\end{rSubsection}

\end{rSection}

%----------------------------------------------------------------------------------------
%   SELECTED PUBLICATIONS 
%----------------------------------------------------------------------------------------
%
\begin{rSection}{Peer-Reviewed Conference Publications}
Uncertainty-aware generative models for inferring document class prevalence.
\textbf{Katherine A. Keith} and Brendan O'Connor. 
In \emph{Proceedings of Empirical Methods in Natural Language Processing (EMNLP)}. 2018. 

Monte Carlo Syntax Marginals for Exploring and Using Dependency Parses.
\textbf{Katherine A. Keith}, Su Lin Blodgett, and Brendan O'Connor.
In \emph{Proceedings of North American Chapter of the Association for Computational Linguistics} (NAACL). 2018.

Identifying civilians killed by police with distantly supervised entity-event extraction. 
\textbf{Katherine A. Keith}, Abram Handler, Michael Pinkham, Cara Magliozzi, Joshua McDuffie, and Brendan O'Connor. In \emph{Proceedings of Empirical Methods in Natural Language Processing (EMNLP)}. 2017. 

\end{rSection}

%----------------------------------------------------------------------------------------
%	INDUSTRY EXPERIENCE SECTION
%----------------------------------------------------------------------------------------

\begin{rSection}{Industry Experience}

\begin{rSubsection}{Research Intern}{May--August 2018}{CTO Data Science Team, Bloomberg L.P.}{New York, New York}
\item Mentor: Dr. Amanda Stent
\item Correlated stock market signals with text of earnings call transcripts 
\end{rSubsection}

\end{rSection}

%----------------------------------------------------------------------------------------
%	TEACHING EXPERIENCE SECTION
%----------------------------------------------------------------------------------------

\begin{rSection}{Teaching Experience}

\begin{rSubsection}{Instructor, First-year seminar}{Fall 2019}{University of Massachusetts Amherst}{}
\item Co-designed curriculum on ``Ethical Issues Surrounding Artificial Intelligence Systems and Big Data" \url{https://github.com/sblodgett/ai-ethics}
\item Led two weekly discussion sections comprising of 19 students each 
\end{rSubsection}

\begin{rSubsection}{Graduate Teaching Assistant \\ CS685: Advanced Natural Language Processing}{Spring 2018}{University of Massachusetts Amherst}{}
\item Assisted students with course material and homework during weekly office hours
\item Graded literature review assignment and in-class presentations 
\item Helped to design homeworks
\end{rSubsection}

\begin{rSubsection}{Fulbright English Teaching Assistant}{August 2015--June 2016}{U.S. Department of State}{Kinmen, Taiwan}
\item Taught first through sixth grade ESL courses in a public elementary school
\item Facilitated multi-cultural dialogue and programming
\end{rSubsection}

\begin{rSubsection}{Mathematics Tutor}{January 2012--May 2015}{Lewis \& Clark College}{Portland, OR}
\item Tutored Calculus I, Calculus II, and Linear Algebra for private one-on-one and group sessions
\item Tutored in the Symbolic and Quantitative Resource Center (SQRC) 
\end{rSubsection}

\end{rSection}

%----------------------------------------------------------------------------------------
%	PRESENTATIONS AND POSTERS  
%----------------------------------------------------------------------------------------

\begin{rSection}{Other Projects}

\begin{itemize}
\item ``Fairkit-learn:  A multi-objective optimization approach to fairness in machine learning classifiers." \emph{Advanced Software Engineering: Analysis and Evaluation} final class project (Spring 2018)
\item ``Class-conditional language modeling with LSTMs." \emph{Machine Learning} final class project (Fall 2017)
\item ``Linguistically Motivated LSTM Architectures for Relation Extraction." \emph{Neural Networks} final class project (Fall 2017)  
\item ``Temporal, Embedding-Based Soft Deduplication of Police Killing Events." \emph{Database Design \& Implementation} final class project (Spring 2017) 
\item  ``Probabilistic Modeling of Trending Words on Twitter." \emph{Statistical Machine Learning} final class project (Fall 2016) 
\item ``Machine Learning Classification of Job Loss Twitter Messages." \emph{Introduction to Natural Language Processing} final class project (Fall 2016) 
\item ``Extending the Pontryagin Maximum Principle of Optimal Control Theory for Inequality Constraints and Discounting." \emph{Lewis \& Clark College Mathematics Department Senior Honors Thesis} (Spring 2015)
\item ``An Agent-Based Simulation of Intergenerational Mobility Amongst the English Medieval Peasantry." \emph{Andrew W. Mellon Student-Faculty Research Project} (Summer 2014) 
\end{itemize} 

\end{rSection}


%----------------------------------------------------------------------------------------
%	SELECTED COURSES 
%----------------------------------------------------------------------------------------

\begin{rSection}{Selected Courses}
Machine Learning, Neural Networks, Probabilistic Graphical Models, Advanced Algorithms and Analysis, Advanced Software Engineering: Analysis and Evaluation, Database Design \& Implementation, Introduction to Natural Language Processing, Advanced Probability \& Statistics, Real Analysis, Abstract Algebra, Game Theory, Numerical Analysis, Differential Equations, Linear Algebra 
\end{rSection}

%----------------------------------------------------------------------------------------
%	SERVICE 
%----------------------------------------------------------------------------------------

\begin{rSection}{Service \& Outreach}

\begin{rSubsection}{Reviewer}{}{}{}
\item ICWSM, 2019
\end{rSubsection}

\begin{rrSubsection}{Organizer, CICS Male Ally Workshop Series}{2017--2018}{University of Massachusetts Amherst \\}
{\emph{\url{https://github.com/mrlucasch/cics-male-allyship-workshops}}}
\end{rrSubsection}


\begin{rrSubsection}{Student Volunteer, Girls~Inc.~Eureka!~Summer~Workshop}{August 1 \& 3, 2017}{University of Massachusetts Amherst}{}
\end{rrSubsection}

\begin{rrSubsection}{Mentor, Research Experience for Undergraduates (REU)}{Summer 2017}{University of Massachusetts Amherst, College of Information and Computer Science}{}
\end{rrSubsection}

\begin{rrSubsection}{Social Co-Chair,  Computer Science Women's Group}{January 2017--Present}
{University of Massachusetts Amherst Computer Science Women's Group}{}
\end{rrSubsection}

\begin{rrSubsection}{Student Volunteer, Women in Engineering and Computing Career Day}{October 24, 2016}
{University of Massachusetts Amherst \\
}{}
\end{rrSubsection}

\end{rSection}

%----------------------------------------------------------------------------------------
%	TECHNICAL STRENGTHS SECTION
%----------------------------------------------------------------------------------------

\begin{rSection}{Technical Strengths}
Python (scipy, scikit-learn, numpy, pandas), Pytorch 

%\begin{tabular}{ @{} >{\bfseries}l @{\hspace{6ex}} l }

%Machine learning modeling & CRF's, Logistic Regression, SVM's, Gibbs sampling, EM algorithm \\
%Protocols \& APIs & XML, JSON, SOAP, REST \\
%Primary programming language & Python\\
%Deep learning library & Pytorch \\
%Python modules & scipy, scikit-learn, numpy, pandas \\
%Databases & SQL, Postgres\\
%Deep learning libraries & Tensorflow, Pytorch \\
%Development Tools & sublime, emacs, tmux \\
%\end{tabular}

\end{rSection}

%----------------------------------------------------------------------------------------
%	FOREIGN LANGUAGES 
%----------------------------------------------------------------------------------------

\begin{rSection}{Foreign Languages}
\textbf{Chinese (Mandarin)} \\
HSK Level 4 (tested April 16, 2016) \\
CET Beijing: 16-week language-intensive immersion program (Spring 2014)
\end{rSection}

%----------------------------------------------------------------------------------------
%	AWARDS SECTION
%----------------------------------------------------------------------------------------

\begin{rSection}{Awards \& Honors}
\begin{itemize}
\item Computing Research Association of Women (CRAW) Graduate Cohort (2017, 2018) 
\item Empirical Methods in Natural Language Processing (EMNLP) Student Travel Scholarship (2017)
\item Paul Utgoff Memorial Graduate Scholarship in Machine Learning (2016) 
\item Fulbright ETA Grantee with the U.S. Department of State (2015--16)
\item Rhodes Scholarship Finalist (2015) 
\item Marshall Scholarship Finalist (2015) 
\item Rena Ratte Award, Lewis \& Clark College (2015) 
\item Robert B. Pamplin Jr. Society Fellow, Lewis \& Clark College (2012--2015)
\item Dean's List, Lewis \& Clark College (2011--2015)
\item Project Pengyou National Leadership Fellow (2014) 
\item Phi Beta Kappa Member (2014--2015)
\item Pi Mu Epsilon Member (2014--2015)
\item Barbara Hirschi Neely Four-Year Full-Tuition Scholarship Recipient, Lewis \& Clark College (2011--2015)
\item NCAA Division III Cross-Country All-Academic (2012) 
\item Lewis \& Clark Cross-Country Four-Year Varsity Letter Winner (2011--2015)
\end{itemize} 
\end{rSection}
%\blfootnote{Updated: \today}
\end{document}
