%%%%%%%%%%%%%%%%%%%%%%%%%%%%%%%%%%%%%%%%%
% Medium Length Professional CV
% LaTeX Template
% Version 2.0 (8/5/13)
%
% This template has been downloaded from:
% http://www.LaTeXTemplatses.com
% 
% Original author:
% Trey Hunner (http://www..com/)
%
% Important note:
% This template requires the resume.cls file to be in the same directory as the
% .tex file. The resume.cls file provides the resume style used for structuring the
% document.
%
%%%%%%%%%%%%%%%%%%%%%%%%%%%%%%%%%%%%%%%%%

%----------------------------------------------------------------------------------------
%   PACKAGES AND OTHER DOCUMENT CONFIGURATIONS
%----------------------------------------------------------------------------------------

\documentclass{resume} % Use the custom resume.cls style

\usepackage[left=0.75in,top=0.6in,right=0.75in,bottom=0.6in]{geometry} % Document margins

\usepackage{enumitem}
\setlist{leftmargin=0em}
\renewcommand{\labelitemi}{$\cdot$}

%%MINE THAT I ADDED

\usepackage{lipsum}
\usepackage{datetime}
\usepackage{hyperref}
\usepackage{ragged2e}
\usepackage{etaremune}
\usepackage{changepage} 

\usepackage{amsmath}
\newcommand{\smalldot}{\raisebox{0.2ex}{\tiny$\bullet$}}

\newcommand\blfootnote[1]{%
  \begingroup
  \renewcommand\thefootnote{}\footnote{#1}%
  \addtocounter{footnote}{-1}%
  \endgroup
}

%% Highlighting
\usepackage{soul}
\usepackage{xcolor}

%---------------------------

\name{Katherine A. Keith} % Your name
\address{
%(406)~$\cdot$~539~$\cdot$~3888 \\ 
kak5@williams.edu \\
https://kakeith.github.io/} % Your phone number and email

\begin{document}

%%----------------------------------------------------------------------------------------
%%  RESEARCH INTERESTS
%%----------------------------------------------------------------------------------------
%
%\begin{rSection}{Research Interests}
%Natural language processing, machine learning, computational social science 
%\end{rSection}

%----------------------------------------------------------------------------------------
%   EDUCATION SECTION
%----------------------------------------------------------------------------------------

\begin{rSection}{Education}

{\bf University of Massachusetts Amherst} \hfill {2016--2021} \\ 
Ph.D. in Computer Science, September 2021 \hfill {\em Amherst, MA} \\
M.S. in Computer Science, February 2020
 
% GPA 3.91/4.0 \\

{\bf Lewis \& Clark College} \hfill {2011--2015} \\ 
B.A. in Mathematics with departmental honors, summa cum laude \hfill {\em Portland, OR} \\
Minor: Chinese Language
% GPA 3.95/4.0 \\
\end{rSection}

\begin{rSection}{Academic Appointments}
{\bf Assistant Professor} \hfill {July 2022--Present} \\
Department of Computer Science \hfill {\em Williamstown, MA} \\ 
 Williams College
\end{rSection}

%----------------------------------------------------------------------------------------
% RESEARCH EXPERIENCE SECTION
%----------------------------------------------------------------------------------------

\begin{rSection}{Research Experience}

\begin{rSubsection}{Postdoctoral Young Investigator}{September 2021--June 2022}{Allen Institute for Artificial Intelligence}{Seattle, WA}
\item Research with the Semantic Scholar team
\end{rSubsection}

\begin{rSubsection}{Graduate Research Assistant}{September 2016--August, 2021}{University of Massachusetts, Amherst}{Amherst, MA}
\item Natural language processing, machine learning, and computational social science research
\item Advisor: Dr. Brendan O'Connor 
\end{rSubsection}

\begin{rSubsection}{Research Intern}{June--August 2020}{AI Discovery Team, Bloomberg}{Remote (London, UK team)}
\item Mentors: Dr.~Edgar Meij, Dr.~Christoph Teichmann
%\item Resulted in publication at the \emph{Workshop on Natural Language Processing and Computational Social Science (NLP+CSS) at EMNLP, 2020}.
\end{rSubsection}

\begin{rSubsection}{Research Intern}{May--August 2018}{CTO Data Science Team, Bloomberg}{New York, New York}
\item Mentor: Dr.~Amanda Stent
%\item Resulted in publication at the \emph{Association of Computational Linguistics (ACL), 2019}.
\end{rSubsection}

%------------------------------------------------

% \begin{rSubsection}{Undergraduate Research Assistant (Economics)}{May--August 2014}{Lewis \& Clark College}{Portland, OR}
% \item Developed an agent-based simulation of intergenerational wealth transfer in medieval England
% \item Funding: Andrew W. Mellon Collaborative Student-Faculty Research Grant
% \item Advisor: Dr. Clifford Bekar
% \end{rSubsection}

\end{rSection}

\begin{rSection}{Grants}
\begin{itemize}
%HIGHLIGHT
\item National Science Foundation. CRII: RI: RUI: ``NLP Models in Heterogeneous Causal Effect Estimation.'' Awarded June 2025. (\$169,637). 
%HIGHLIGHT
\item Massachusetts Life Sciences Center's Workforce Development Capital Grant. With Stephen N.~Freund, Noah J.~Sandstrom, and Victor A.~Cazares. Awarded July 2024. (\$744,446 + \$83,200 matching funds)  
\item Young Investigator Grant from the \emph{Allen Institute for Artificial Intelligence}. Awarded November 2022. (\$100,000) 
\item  \emph{Social Science Research Council (SSRC)/Summer Institutes in Computational Social Science (SICSS)} Research Grant. With Ian Stewart. Awarded 2021. (\$1,200)
\item \emph{Kaggle} Open Data Research Grant. Awarded 2020. With Andy Halterman and Sheikh Sarwar. (\$5,000)
\end{itemize}

\end{rSection}

\newpage
\begin{rSection}{Awards \& Honors}
\begin{itemize}
\item Bloomberg Data Science Ph.D. Fellowship (2019-2021)  
\item Outstanding Reviewer, EMNLP 2020
\item Awarded \emph{distinction} for Ph.D. candidacy portfolio submitted to the College of Information and Computer Sciences at University of Massachusetts Amherst (December 2019) 
\item Empirical Methods in Natural Language Processing (EMNLP) Student Travel Scholarship (2017)
\item Paul Utgoff Memorial Graduate Scholarship in Machine Learning (2016) 
\item Rena J.~Ratte Memorial Award, Lewis \& Clark College (2015)-- \emph{``This is the highest academic award given at Lewis \& Clark and is given annually to one undergraduate of senior standing whose abilities and commitment have combined to produce work which is consistently of the greatest distinction.''} 
\item NCAA Division III Cross-Country All-Academic (2012) 
\end{itemize} 
\end{rSection}


%----------------------------------------------------------------------------------------
%   SELECTED PUBLICATIONS 
%----------------------------------------------------------------------------------------
%
%----------------------------------------------------------------------------------------
%   INDUSTRY EXPERIENCE SECTION
%----------------------------------------------------------------------------------------
\begin{rSection}{Journal Publications (Peer-Reviewed)}
\begin{etaremune}

%HIGHLIGHT
\item ``Let Me Just Interrupt You'': Estimating Gender Effects in Supreme Court Oral Arguments. 
Erica Cai, Ankita Gupta, \textbf{Katherine A. Keith}, Brendan O'Connor, and Douglas R. Rice ($^*$\emph{authors alphabetical by last name}). 
\emph{Journal of Law and Courts}, Cambridge University Press. March, 2025. 

\item RCT Rejection Sampling for Causal Estimation Evaluation. 
\textbf{Katherine A. Keith}, Sergey Feldman, David Jurgens, Jonathan Bragg, and Rohit Bhattacharya.
\emph{Transactions on Machine Learning Research (TMLR)}. 2023.  


\item Causal Inference in Natural Language Processing: Estimation, Prediction, Interpretation and Beyond. Amir Feder, \textbf{Katherine A. Keith}, Emaad Manzoor, Reid Pryzant, Dhanya Sridhar, Zach Wood-Doughty, Jacob Eisenstein, Justin Grimmer, Roi Reichart, Margaret E. Roberts, Brandon M. Stewart, Victor Veitch, Diyi Yang.
\emph{Transactions of the Association for Computational Linguistics (TACL)}. 2022.  

\end{etaremune}
\end{rSection}

\begin{rSection}{Conference Proceedings Publications (Peer-Reviewed)}
\begin{etaremune}

%HIGHLIGHT
\item Proximal Causal Inference With Text Data.
    Jacob M. Chen, Rohit Bhattacharya, and \textbf{Katherine A.~Keith}.
    In  Conference on \emph{Neural Information Processing Systems (NeurIPS)}. 2024. 

\item Causal Matching with Text Embeddings: A Case Study in Estimating the Causal Effects of Peer Review Policies.
    Raymond Z. Zhang, Neha Nayak Kennard, 
    Daniel Scott Smith, Daniel A. McFarland, 
    Andrew McCallum, and \textbf{Katherine A. Keith}.
    In \emph{Findings of the Association for Computational Linguistics (ACL Findings)}. 2023.

\item Words as Gatekeepers: Measuring Discipline-specific
    Terms and Meanings in Scholarly Publications.
    Li Lucy, Jesse Dodge, David Bamman, and \textbf{Katherine A. Keith}.
    In \emph{Findings of the Association for Computational Linguistics (ACL Findings)}. 2023.

%highlight 
\item Paying Attention to the Algorithm Behind the Curtain: Bringing Transparency to YouTube's Demonetization Algorithms.
    Arun Dunna, \textbf{Katherine A. Keith}, Ethan Zuckerman, Narseo Vallina-Rodriguez, Brendan O'Connor, Rishab Nithyanand. 
    In \emph{ACM Conference On Computer-Supported Cooperative Work And Social Computing (CSCW)}. 2022. 

\item Corpus-Level Evaluation for Event QA: The IndiaPoliceEvents Corpus Covering the 2002 Gujarat Violence. Andrew Halterman$^*$, \textbf{Katherine A. Keith}$^*$, Sheikh Muhammad Sarwar$^*$, and Brendan O'Connor ($^*$ indicates joint first-authors).  In \emph{Findings of the Association for Computational Linguistics (ACL Findings)}. 2021.

\item Text and Causal Inference: A Review of Using Text to Remove Confounding from Causal Estimates. \textbf{Katherine A. Keith}, David Jensen, and Brendan O'Connor. In \emph{Proceedings of Annual Meeting of the Association for Computational Linguistics (ACL)}. 2020.  

\item Modeling financial analysts' decision making via the pragmatics and semantics of earnings calls. 
\textbf{Katherine A. Keith} and Amanda Stent. 
In \emph{Proceedings of Annual Meeting of the Association for Computational Linguistics (ACL)}.  2019. 

\item Uncertainty-aware generative models for inferring document class prevalence.
\textbf{Katherine A. Keith} and Brendan O'Connor. 
In \emph{Proceedings of Empirical Methods in Natural Language Processing (EMNLP)}. 2018. 

\item Monte Carlo Syntax Marginals for Exploring and Using Dependency Parses.
\textbf{Katherine A. Keith}, Su Lin Blodgett, and Brendan O'Connor.
In \emph{Proceedings of North American Chapter of the Association for Computational Linguistics} (NAACL). 2018.

\item Identifying civilians killed by police with distantly supervised entity-event extraction. 
\textbf{Katherine A. Keith}, Abram Handler, Michael Pinkham, Cara Magliozzi, Joshua McDuffie, and Brendan O'Connor. In \emph{Proceedings of Empirical Methods in Natural Language Processing (EMNLP)}. 2017. 

\end{etaremune}
\end{rSection}

\begin{rSection}{Workshop Publications (Lightly Peer-Reviewed)}
\begin{etaremune}

\item Democratizing Machine Learning for Interdisciplinary Scholars:
Reflections on the NLP+CSS Tutorial Series. 
\textbf{Katherine A.~Keith} and Ian Stewart. 
In \emph{Proceedings of the 1st Workshop on Teaching for NLP}. 2023. 

\item Literary Intertextual Semantic Change Detection:
Application and Motivation for Evaluating Models on Small Corpora.
Jackson Ehrenworth and \textbf{Katherine A.~Keith}. 
In \emph{Proceedings of the 4th Workshop on Computational Approaches to Historical Language Change at EMNLP}. 2023. 


\item Text as Causal Mediators: Research Design for Causal Estimates of Differential Treatment of Social Groups via Language Aspects. \textbf{Katherine A. Keith}, Douglas Rice, and Brendan O'Connor. In \emph{Proceedings of the First Workshop on Causal Inference and Natural Language Processing (CI+NLP) at EMNLP}. 2021.  

\item Uncertainty over Uncertainty: Investigating the Assumptions, Annotations, and Text Measurements of Economic Policy Uncertainty. \textbf{Katherine A. Keith}, Christoph Teichmann, Brendan O’Connor, and Edgar Meij.  In \emph{Proceedings of the Fourth Workshop on Natural Language Processing and Computational Social Science (NLP+CSS) at EMNLP}. 2020.

\end{etaremune}
\end{rSection}

%----------------------------------------------------------------------------------------
%   TEACHING EXPERIENCE SECTION
%----------------------------------------------------------------------------------------

\begin{rSection}{Teaching Experience}

\textbf{Instructor}

\begin{itemize}
    % HIGHLIGHT
    \item CSCI 136: Data Structures \& Advanced Programming (Williams College, Spring 2025)
    \item[] ~~~2 lecture sections, 4 lab sections, 52 students, 11 TAs

    % HIGHLIGHT
    \item CSCI 375: Natural Language Processing (Williams College, Fall 2024)
    \item[] ~~~1 lecture section, 26 students, 1 TA

    \item CSCI 136: Data Structures \& Advanced Programming (Williams College, Spring 2024)
    \item[] ~~~2 lecture sections, 4 lab sections, 59 students, 9 TAs

    \item CSCI 104: Data Science and Computing For All (Williams College, Fall 2023; Co-instructor with Dr. Stephen Freund)
    \item[] ~~~2 instructors splitting: 2 lecture sections, 4 lab sections, 61 students, 8 TAs

    \item CSCI 375: Natural Language Processing (Williams College, Spring 2023)
    \item[] ~~~2 lecture sections, 46 students, 2 TAs

    \item CSCI 104: Understanding Data Through Computation (Williams College, Fall 2022; Co-instructor with Dr. Stephen Freund)
    \item[] ~~~2 instructors splitting: 2 lecture sections, 4 lab sections, 48 students, 8 TAs

    \item CS335: Machine Learning (Mount Holyoke College, Spring 2020)
    \item[] ~~~1 lecture section, 19 students

    \item CS191: Ethical Issues Surrounding Artificial Intelligence Systems and Big Data, First Year Seminar (University of Massachusetts Amherst, Fall 2019)
\end{itemize}


% \begin{itemize}
%   \item CSCI 136: Data Structures \& Advanced Programming (Williams College, Spring 2024) 

%   \item CSCI 104: Data Science and Computing For All (Williams College, Fall 2023)
%   %\item CSCI 104: Data Science and Computing For All (Williams College, Fall 2023)
%   \item CSCI 375: Natural Language Processing (Williams College, Spring 2023)
%   \item CSCI 104: Understanding Data Through Computation (Williams College, Fall 2022)
%   \item CS335: Machine Learning (Mount Holyoke College, Spring 2020)
%   \item CS191: Ethical Issues Surrounding Artificial Intelligence Systems and Big Data, First Year Seminar (University of Massachusetts Amherst, Fall 2019)

% \end{itemize}

\textbf{Graduate Teaching Assistant}
\begin{itemize}
\item CS685: Advanced Natural Language Processing (University of Massachusetts Amherst, Spring 2018)
\end{itemize}

% % \begin{rSubsection}
% % {Instructor \\}{}{Williams College}{}
% %   \item CSCI 104: Understanding Data Through Computation (Fall 2023)
% % \end{rSubsection}

% \begin{rSubsection}
% {Instructor\\
% CS335: Machine Learning}
% {Spring 2020}
% {Mount Holyoke College}{}
% \item Sole instructor for 19 students. Designed and delivered lectures, held office hours, developed problem sets, designed exams, and graded problem sets and exams.

% \end{rSubsection}

% \begin{rSubsection}{Instructor \\CS191: Computer Science First Year Seminar}{Fall 2019}{University of Massachusetts Amherst}{}
% \item Co-designed curriculum on ``Ethical Issues Surrounding Artificial Intelligence Systems and Big Data.''
% \item Led two weekly discussion sections comprising of 19 students each. 
% \end{rSubsection}

% \begin{rSubsection}{Graduate Teaching Assistant \\ CS685: Advanced Natural Language Processing}{Spring 2018}{University of Massachusetts Amherst}{}
% \item Assisted students with course material and homework during weekly office hours.
% \item Co-designed homework assignments, graded literature review assignment and in-class presentations. 
% \end{rSubsection}

\begin{rSubsection}{Fulbright English Teaching Assistant}{August 2015--June 2016}{U.S. Department of State}{Kinmen, Taiwan}
\item Taught first through sixth grade ESL courses in a public elementary school. Facilitated multi-cultural dialogue and events.
\end{rSubsection}

\begin{rSubsection}{Tutorials and Workshops}{}{}{}
%HIGHLIGHT 
\item Python Data Wrangling with AI Co-Pilots. Williams College. June 18-19, 2025. 
\item \emph{Aggregated Classification Pipelines: Propagating Probabilistic Assumptions from Start to Finish.} NLP+CSS 201 Online Tutorial Series. March 30, 2022. 
\end{rSubsection}

% \begin{rSubsection}{Mathematics Tutor}{January 2012--May 2015}{Lewis \& Clark College}{Portland, OR}
% \item Tutored students in Calculus I, Calculus II, and Linear Algebra.
% \item Tutored in the Symbolic and Quantitative Resource Center (SQRC). 
% \end{rSubsection}

\end{rSection}

%----------------------------------------------------------------------------------------
%   MENTEES SECTION
%----------------------------------------------------------------------------------------

\begin{rSection}{Student Research Mentoring}
% During summer research experience for undergraduates (REU), I met with students daily to guide them through technical research challenges. When advising an undergraduate honors thesis, I met with the student for one hour weekly to address any technical issues and outline future work. 

\textbf{Undergraduate Honors Thesis Advisees}
\begin{itemize}[leftmargin=0.5cm]
\item Alisa Kanganis, AY 2024-2025 -- \emph{Awarded High Honors}
\end{itemize}

\textbf{Student Research Mentees}
\begin{itemize}[leftmargin=0.5cm]
\item Himal Pandey (Winter Study 2025)
\item Jeeyon Kang (Research Assistant, Fall 2024; Winter Study 2025)
\item Jacob Chen (Undergraduate Research, Summer 2023; Post-baccalaureate Research Assistant, Spring 2024)
\item Justin Cheigh (Research Independent Study, Spring 2024)
\item Angus Li (Undergraduate Research, Summer 2023; Research Assistant Fall 2023)

\item Serah Park (Undergraduate Research, Summer 2023)
\item Lucy Li (PhD intern at AI2, 2022) -- \emph{Winner of AI2's Outstanding Intern of the Year Award}
\item Sirius Just (Undergraduate Honors Thesis, AY 2019-2020; REU, Summer 2019) 
\item Tamas Palfi (REU, Summer 2019) 
\item Hieu Phan (REU, Summer 2019) 
\item Aaron Mueller (REU, Summer 2017) 
\end{itemize}


\textbf{Undergraduate Honors Thesis Committee Member (Second Reader)}

    \begin{itemize}[leftmargin=0.5cm]
    %HIGHLIGHT
    \item Matt Laws (AY 2024-2025)
    \item Nathaniel Tunggal (AY 2024-2025)
    \item Max Ellis (AY 2023-2024)
    \item Elijah Tamarchenko (AY 2022-2023)
    \end{itemize}

\textbf{PhD Dissertation Committee Member}

\begin{itemize}[leftmargin=0.5cm]
    \item Tanise Ceron. University of Stuttgart, Germany. October 14, 2024. 
\end{itemize}

\end{rSection}

%----------------------------------------------------------------------------------------
%   INVITED TALKS 
%----------------------------------------------------------------------------------------
\begin{rSection}{Speaking Engagements}

\begin{rSubsection}{Invited Talks}{}{}{}
%HIGHLIGHT
\item ``Proximal Causal Inference with Text Data''

\begin{adjustwidth}{2em}{0pt}
\begin{itemize}
    \item Seminar Series of the Data Science and Artificial Intelligence Division (DSAI) at the Chalmers University of Technology (online) ---January 20, 2025. 
    \item One-U Responsible AI Initiative, University of Utah. January, 27, 2025. 
    \item Online Causal Inference Seminar --- March 11, 2025.  
\end{itemize}
\end{adjustwidth}

\item University of Stuttgart. ` ``Let Me Just Interrupt You":
Estimating Gender Effects in Supreme
Court Oral Arguments.'---October 14, 2024.

\item ``Text as a High-Dimensional Causal Confounding Variable: An Empirical Framework for Evaluation''

\begin{adjustwidth}{2em}{0pt}
\begin{itemize}
    \item  Joint Statistical Meeting (JSM) ---August 5, 2024
    \item American Causal Inference Conference, ``Statistical Innovations in Causal Inference with Text as Data: Emerging Trends and Future Directions" invited session---May 16, 2024.
\end{itemize}
\end{adjustwidth}

\item ``Melding NLP and Causal Inference''

\begin{adjustwidth}{2em}{0pt}
\begin{itemize}
    \item  University of Iowa, Computer Science Colloquium (online) --- September 8, 2023.
    \item Institute for Analytical Sociology (online) --- February 8, 2023. 
\end{itemize}
\end{adjustwidth}

\item Discussant, New Directions in Analyzing Text as Data (TADA), 2023. 
\item Williams College Statistics Colloquium ``Automated Event Extraction for
News-Based Counterdata''---October 19, 2022.
\item Williams College Computer Science Colloquium---November 15, 2019
\item Rutgers University's Natural Language Processing Group---November 6, 2019
\item Lewis \& Clark College's Mathematical Sciences Colloquium---September 24, 2018
\end{rSubsection}

\begin{rSubsection}{Guest Lectures}{}{}{}
\item Williams College, ECON 460: Women, Work, and the World Economy. ``Let Me Just Interrupt You: Estimating Gender Effects in Supreme Court Oral Arguments."---November 13, 2024.
\item Stanford University, CS 293 / EDUC 473: Empowering Educators via Language Technology.``Causal Effect Estimation with Text Data''--- October 25, 2023. 
\item University of Massachusetts Amherst, COMPSCI 692L: Seminar, Natural Language Processing, ``RCT Rejection Sampling for Causal Estimation Evaluation'' --- May 2, 2023
\item Georgia Institute of Technology, CS-6471: Computational Social Science, ``Text and Causal Estimation: Text as Causal Confounders and Mediators''---March 28, 2022
\item University of Massachusetts Amherst, Natural Language Processing---November 13, 2020 
\item Mount Holyoke College, Natural Language Processing---November 13, 2019
\item Mount Holyoke College, Natural Language Processing---October 10, 2018
\item University of Massachusetts Amherst, Advanced Natural Language Processing---April 26, 2018
\end{rSubsection}

\end{rSection}

%----------------------------------------------------------------------------------------
%   PRESENTATIONS AND POSTERS  
%----------------------------------------------------------------------------------------

% \begin{rSection}{Other Projects}

%  \begin{itemize}
%  \item ``Fairkit-learn:  A multi-objective optimization approach to fairness in machine learning classifiers." With Sam Witty. \emph{Advanced Software Engineering: Analysis and Evaluation} final class project (Spring 2018) 
%  \item ``Class-conditional language modeling with LSTMs." \emph{Machine Learning} final class project (Fall 2017)
%  \item ``Linguistically Motivated LSTM Architectures for Relation Extraction." With Sheshera Mysore. \emph{Neural Networks} final class project (Fall 2017)  
%  \item ``Temporal, Embedding-Based Soft Deduplication of Police Killing Events." With Su Lin Blodgett and Aaron Schein. \emph{Database Design \& Implementation} final class project (Spring 2017)  
%  \item  ``Probabilistic Modeling of Trending Words on Twitter." With Su Lin Blodgett.  \emph{Statistical Machine Learning} final class project (Fall 2016)  
%  \item ``Machine Learning Classification of Job Loss Twitter Messages." With Boya Ren. \emph{Introduction to Natural Language Processing} final class project (Fall 2016) 
%  \item ``Extending the Pontryagin Maximum Principle of Optimal Control Theory for Inequality Constraints and Discounting." \emph{Lewis \& Clark College Mathematics Department Senior Honors Thesis} (Spring 2015)
%  \item ``An Agent-Based Simulation of Intergenerational Mobility Amongst the English Medieval Peasantry." \emph{Andrew W. Mellon Student-Faculty Research Project} (Summer 2014) 
%  \end{itemize} 

% \end{rSection}

%----------------------------------------------------------------------------------------
%   SELECTED COURSES 
%----------------------------------------------------------------------------------------

% \begin{rSection}{Selected Courses}
% Machine Learning, Neural Networks, Probabilistic Graphical Models, Advanced Algorithms and Analysis, Advanced Software Engineering: Analysis and Evaluation, Database Design \& Implementation, Introduction to Natural Language Processing, Advanced Probability \& Statistics, Real Analysis, Abstract Algebra, Game Theory, Numerical Analysis, Differential Equations, Linear Algebra 
% \end{rSection}

%----------------------------------------------------------------------------------------
%   SERVICE 
%----------------------------------------------------------------------------------------

\begin{rSection}{Service \& Outreach}

\begin{rSubsection}{Williams College Service}{}{}{}
\item Science Executive Committee, At-large Junior Faculty Member, AY 2024-2025
\item Colloquium Organizer, Computer Science Department, AY 2023-2024
\item Women in Computer Science (WiCS) Organizer, Computer Science Department, AY 2023-2024, AY 2024-2025
\item Underrepresented Identities in Computer Science (UniCS) Faculty Mentor, Computer Science Department, AY 2022-2023, AY 2024-2025
\end{rSubsection}

\begin{rSubsection}{Leadership Roles in the Research Community}{}{}{}
% highlight\
\item Co-organizer, \emph{NLP+CSS} workshop at NAACL 2024.  June 21, 2024.
\item Area Chair, \emph{Computational Social Science and Cultural Analytics Track}, EMNLP 2023.
\item Organizing Committee/Program Committee. New Directions in Analyzing Text as Data (TADA), 2023.
\item Co-organizer, \emph{NLP+CSS} workshop at EMNLP. December 7, 2022. 
\item Co-organizer, \emph{NLP+CSS 201: Beyond the Basics} online tutorial series (Fall 2021)
\item Co-organizer and Panel Moderator, \emph{First Workshop on Causal Inference and Natural Language Processing} at EMNLP 2021. November 10, 2021.
\end{rSubsection}

\noindent
\begin{rSubsection}{Reviewer / Referee}{}{}{}
\item ACL Rolling Review (ARR): Feburary 2025, May 2025
\item Ethics Review, ACL Rolling Review (ARR) Ethics: Feburary, 2025
\item CLeaR: 2025
\item NeurIPS: 2024
\item NSF~Ad-Hoc~Reviewer: 2024
\item TACL: 2023
\item TADA: 2022, 2023
\item Journal of Data Science: February, 2022
\item Science Advances: October, 2021
\item Workshop on Teaching NLP: 2021
\item ACL: 2020, 2021, 2023
\item NAACL: 2019, 2021
\item EACL: 2021 
\item CoNLL: 2020 
\item EMNLP: 2020 
\item ICWSM: 2019, 2020
\end{rSubsection}

\begin{rSubsection}{Other Outreach Roles}{}{}{}
\item Panelist, Teaching at Teaching Intensive Institutions, University of Massachusetts Amherst (October 20, 2023)
\item Panelist, University of Massachusetts CSWomen panel on ``Applying to Research Internships and Jobs'' (November 19, 2021)
\item Panelist, NSF CSGrad4US Mentoring Program panel on ``What I wish I knew when I started graduate school'' (November 24, 2021)
\item 
{PhD Applicant Support Mentor for Underrepresented Students} ({November 2020}) {University of Massachusetts Amherst}
\item 
{Panelist, CS Graduate School Q\&A Panel} ({May 1, 2020}) {Cape Cod Community College}

\item 
{Co-chair, CSWomen} ({February~2019--January 2020}) {University of Massachusetts Amherst Computer Science Women's Group}

\item 
{Social Co-Chair,  CSWomen} ({January 2017--January 2019}) 
{University of Massachusetts Amherst Computer Science Women's Group}

\item {Student Volunteer, Women in Engineering and Computing Career Day} ({October 28, 2019})
{University of Massachusetts Amherst}{}


\item {``What is computer science?" 7th Grade Guest Lecture} ({Jan. 3, 2018 and Jan. 7, 2019}) 
{Chief Joseph Middle School, Bozeman, Montana}

\item {Organizer, CICS Graduate Male Ally Workshop Series} ({September 2017-- May 2018}) {University of Massachusetts Amherst}

\item {Student Volunteer, Girls~Inc.~Eureka!~Summer~Workshop} ({August 1 \& 3, 2017}) {University of Massachusetts Amherst}

\item {Student Volunteer, Women in Engineering and Computing Career Day} ({October 24, 2016})
{University of Massachusetts Amherst}{}
\end{rSubsection}

\end{rSection}


%----------------------------------------------------------------------------------------
%   TECHNICAL STRENGTHS SECTION
%----------------------------------------------------------------------------------------

%    \begin{rSection}{Technical Strengths}
%    Python (scipy, scikit-learn, numpy, pandas), Pytorch, R, SQL  

%\begin{tabular}{ @{} >{\bfseries}l @{\hspace{6ex}} l }

%Machine learning modeling & CRF's, Logistic Regression, SVM's, Gibbs sampling, EM algorithm \\
%Protocols \& APIs & XML, JSON, SOAP, REST \\
%Primary programming language & Python\\
%Deep learning library & Pytorch \\
%Python modules & scipy, scikit-learn, numpy, pandas \\
%Databases & SQL, Postgres\\
%Deep learning libraries & Tensorflow, Pytorch \\
%Development Tools & sublime, emacs, tmux \\
% Data visualization: matplotlib, omnigraffle, ggplot2
%\end{tabular}

%\end{rSection}

%----------------------------------------------------------------------------------------
%   AWARDS SECTION
%----------------------------------------------------------------------------------------

% \begin{rSection}{Undergraduate Awards \& Honors}
% \begin{itemize}
% \item Fulbright ETA Grantee with the U.S. Department of State (2015--16)
% \item Rhodes Scholarship Finalist (2015) 
% \item Marshall Scholarship Finalist (2015) 
% \item Rena J.~Ratte Memorial Award, Lewis \& Clark College (2015)-- \emph{``This is the highest academic award given at Lewis \& Clark and is given annually to one undergraduate of senior standing whose abilities and commitment have combined to produce work which is consistently of the greatest distinction.''} 
% %\url{https://college.lclark.edu/dean/faculty-and-student-awards/}
% % \item Robert B. Pamplin Jr. Society Fellow, Lewis \& Clark College (2012--2015)
% % \item Dean's List, Lewis \& Clark College (2011--2015)
% % %\item Phi Beta Kappa Member (2014--2015)
% % \item Pi Mu Epsilon Member (2014--2015)
% % \item Barbara Hirschi Neely Four-Year Full-Tuition Scholarship Recipient, Lewis \& Clark College (2011--2015)
% \item NCAA Division III Cross-Country All-Academic (2012) 
% \item Lewis \& Clark College Cross-Country Four-Year Varsity Letter Winner (2011--2015)
% \end{itemize} 
% \end{rSection}

\begin{rSection}{Professional Development}


\begin{rSubsection}{Academic Workshops and Trainings}{}{}{}
\item Inclusive Hiring Strategies for Faculty Search Committees (Williams College). September 13, 2024.
\item Faculty Success Program (12-weeks online; Fall 2023) by National Center for Faculty Development and Diversity (NCFDD).---\emph{Intensive online program over the course of twelve weeks consisting of weekly group calls, weekly homeworks, and one-on-one coaching.}
\item Participated in \emph{Claiming Williams Day} ``Creating an Inclusive Classroom'' Workshop with Dr. Anika Daniels-Osaze. February 2, 2023.
\item Participated in the Computing Research Association (CRA)'s Career Mentoring Workshop. Washington DC, February 24-25, 2022. 
\item Computing Research Association of Women (CRA-W) Graduate Cohort (2017, 2018)  
\end{rSubsection}

\begin{rSubsection}{Roundtables}{}{}{}
\item Williams Rountable Program: Research Discussion Group, Spring 2023. (\$500 stipend)
\end{rSubsection}
\end{rSection}

\begin{rSection}{Foreign Languages}
\textbf{Chinese (Mandarin)} \\
HSK Level 4 (test passed April 16, 2016) \\
CET Beijing: 16-week language-intensive immersion program (Spring 2014)
\end{rSection}

\blfootnote{Updated: \today}

\end{document}
