%%%%%%%%%%%%%%%%%%%%%%%%%%%%%%%%%%%%%%%%%
% Medium Length Professional CV
% LaTeX Template
% Version 2.0 (8/5/13)
%
% This template has been downloaded from:
% http://www.LaTeXTemplates.com
%
% Original author:
% Trey Hunner (http://www.treyhunner.com/)
%
% Important note:
% This template requires the resume.cls file to be in the same directory as the
% .tex file. The resume.cls file provides the resume style used for structuring the
% document.
%
%%%%%%%%%%%%%%%%%%%%%%%%%%%%%%%%%%%%%%%%%

%----------------------------------------------------------------------------------------
%   PACKAGES AND OTHER DOCUMENT CONFIGURATIONS
%----------------------------------------------------------------------------------------

\documentclass{resume} % Use the custom resume.cls style

\usepackage[left=0.75in,top=0.6in,right=0.75in,bottom=0.6in]{geometry} % Document margins

\usepackage{enumitem}
\setlist{leftmargin=0em}
\renewcommand{\labelitemi}{$\cdot$}

%%MINE THAT I ADDED

\usepackage{lipsum}
\usepackage{datetime}
\usepackage{hyperref}
\usepackage{ragged2e}
\usepackage{etaremune}

\newcommand\blfootnote[1]{%
  \begingroup
  \renewcommand\thefootnote{}\footnote{#1}%
  \addtocounter{footnote}{-1}%
  \endgroup
}

%---------------------------

\name{Katherine A. Keith} % Your name
\address{
%(406)~$\cdot$~539~$\cdot$~3888 \\ 
kkeith@cs.umass.edu \\
https://kakeith.github.io/} % Your phone number and email

\begin{document}

%%----------------------------------------------------------------------------------------
%%  RESEARCH INTERESTS
%%----------------------------------------------------------------------------------------
%
%\begin{rSection}{Research Interests}
%Natural language processing, machine learning, computational social science 
%\end{rSection}

%----------------------------------------------------------------------------------------
%   EDUCATION SECTION
%----------------------------------------------------------------------------------------

\begin{rSection}{Education}

{\bf University of Massachusetts Amherst} \hfill {2016--2021} \\ 
Ph.D. in Computer Science, September 2021 \hfill {\em Amherst, MA} \\
M.S. in Computer Science, February 2020
 
% GPA 3.91/4.0 \\

{\bf Lewis \& Clark College} \hfill {2011--2015} \\ 
B.A. in Mathematics with departmental honors, summa cum laude \hfill {\em Portland, OR} \\
Minor: Chinese Language\\
% GPA 3.95/4.0 \\
\end{rSection}

%----------------------------------------------------------------------------------------
% RESEARCH EXPERIENCE SECTION
%----------------------------------------------------------------------------------------

\begin{rSection}{Research Experience}

\begin{rSubsection}{Postdoctoral Young Investigator}{September 2021--Present}{Allen Institute for Artificial Intelligence}{Seattle, WA}
\item Research with the Semantic Scholar team
\end{rSubsection}

\begin{rSubsection}{Graduate Research Assistant}{September 2016--August, 2021}{University of Massachusetts, Amherst}{Amherst, MA}
\item Natural language processing, machine learning, and computational social science research
\item Advisor: Dr. Brendan O'Connor 
\end{rSubsection}

\begin{rSubsection}{Research Intern}{June--August 2020}{AI Discovery Team, Bloomberg}{Remote}
\item Mentors: Dr.~Edgar Meij, Dr.~Christoph Teichmann
\item Resulted in publication at the \emph{Workshop on Natural Language Processing and Computational Social Science (NLP+CSS) at EMNLP, 2020}.
\end{rSubsection}

\begin{rSubsection}{Research Intern}{May--August 2018}{CTO Data Science Team, Bloomberg}{New York, New York}
\item Mentor: Dr.~Amanda Stent
\item Resulted in publication at the \emph{Association of Computational Linguistics (ACL), 2019}.
\end{rSubsection}

%------------------------------------------------

\begin{rSubsection}{Undergraduate Research Assistant (Economics)}{May--August 2014}{Lewis \& Clark College}{Portland, OR}
\item Developed an agent-based simulation of intergenerational wealth transfer in medieval England
\item Funding: Andrew W. Mellon Collaborative Student-Faculty Research Grant
\item Advisor: Dr. Clifford Bekar
\end{rSubsection}

\end{rSection}

%----------------------------------------------------------------------------------------
%   SELECTED PUBLICATIONS 
%----------------------------------------------------------------------------------------
%
%----------------------------------------------------------------------------------------
%   INDUSTRY EXPERIENCE SECTION
%----------------------------------------------------------------------------------------


\begin{rSection}{Publications}
\begin{etaremune}
\item Text as Causal Mediators: Research Design for Causal Estimates of Differential Treatment of Social Groups via Language Aspects. \textbf{Katherine A. Keith}, Douglas Rice, and Brendan O'Connor. In \emph{Proceedings of the First Workshop on Causal Inference and Natural Language Processing (CI+NLP) at EMNLP}. 2021.  

\item Corpus-Level Evaluation for Event QA: The IndiaPoliceEvents Corpus Covering the 2002 Gujarat Violence. Andrew Halterman$^*$, \textbf{Katherine A. Keith}$^*$, Sheikh Muhammad Sarwar$^*$, and Brendan O'Connor ($^*$ indicates joint first-authors).  In \emph{Findings of the Association for Computational Linguistics (ACL-IJCNLP)}. 2021.

\item Uncertainty over Uncertainty: Investigating the Assumptions, Annotations, and Text Measurements of Economic Policy Uncertainty. \textbf{Katherine A. Keith}, Christoph Teichmann, Brendan O’Connor, and Edgar Meij.  In \emph{Proceedings of the Fourth Workshop on Natural Language Processing and Computational Social Science (NLP+CSS) at EMNLP}. 2020.

\item Text and Causal Inference: A Review of Using Text to Remove Confounding from Causal Estimates. \textbf{Katherine A. Keith}, David Jensen, and Brendan O'Connor. In \emph{Proceedings of Annual Meeting of the Association for Computational Linguistics (ACL)}. 2020.  

\item Modeling financial analysts' decision making via the pragmatics and semantics of earnings calls. 
\textbf{Katherine A. Keith} and Amanda Stent. 
In \emph{Proceedings of Annual Meeting of the Association for Computational Linguistics (ACL)}.  2019. 

\item Uncertainty-aware generative models for inferring document class prevalence.
\textbf{Katherine A. Keith} and Brendan O'Connor. 
In \emph{Proceedings of Empirical Methods in Natural Language Processing (EMNLP)}. 2018. 

\item Monte Carlo Syntax Marginals for Exploring and Using Dependency Parses.
\textbf{Katherine A. Keith}, Su Lin Blodgett, and Brendan O'Connor.
In \emph{Proceedings of North American Chapter of the Association for Computational Linguistics} (NAACL). 2018.

\item Identifying civilians killed by police with distantly supervised entity-event extraction. 
\textbf{Katherine A. Keith}, Abram Handler, Michael Pinkham, Cara Magliozzi, Joshua McDuffie, and Brendan O'Connor. In \emph{Proceedings of Empirical Methods in Natural Language Processing (EMNLP)}. 2017. 
\end{etaremune}

\end{rSection}



%----------------------------------------------------------------------------------------
%   TEACHING EXPERIENCE SECTION
%----------------------------------------------------------------------------------------

\begin{rSection}{Teaching Experience}

\begin{rSubsection}
{Instructor\\
CS335: Machine Learning}
{Spring 2020}
{Mount Holyoke College}{}
\item Sole instructor for 19 students. Designed and delivered lectures, held office hours, developed problem sets, designed exams, and graded problem sets and exams.

\end{rSubsection}

\begin{rSubsection}{Instructor \\CS191: Computer Science First Year Seminar}{Fall 2019}{University of Massachusetts Amherst}{}
\item Co-designed curriculum on ``Ethical Issues Surrounding Artificial Intelligence Systems and Big Data.''
\item Led two weekly discussion sections comprising of 19 students each. 
\end{rSubsection}

\begin{rSubsection}{Graduate Teaching Assistant \\ CS685: Advanced Natural Language Processing}{Spring 2018}{University of Massachusetts Amherst}{}
\item Assisted students with course material and homework during weekly office hours.
\item Co-designed homework assignments, graded literature review assignment and in-class presentations. 
\end{rSubsection}

\begin{rSubsection}{Fulbright English Teaching Assistant}{August 2015--June 2016}{U.S. Department of State}{Kinmen, Taiwan}
\item Taught first through sixth grade ESL courses in a public elementary school.
\item Facilitated multi-cultural dialogue and events.
\end{rSubsection}

% \begin{rSubsection}{Mathematics Tutor}{January 2012--May 2015}{Lewis \& Clark College}{Portland, OR}
% \item Tutored students in Calculus I, Calculus II, and Linear Algebra.
% \item Tutored in the Symbolic and Quantitative Resource Center (SQRC). 
% \end{rSubsection}

\end{rSection}

%----------------------------------------------------------------------------------------
%   MENTEES SECTION
%----------------------------------------------------------------------------------------

\begin{rSection}{Mentoring Undergraduate Students}
During summer research experience for undergraduates (REU), I met with students daily to guide them through technical research challenges. When advising an undergraduate honors thesis, I met with the student for one hour weekly to address any technical issues and outline future work. 

\begin{itemize}
\item Sirius Just (Undergraduate Honors Thesis, 2019-2020; REU, Summer 2019) 
\item Tamas Palfi (REU, Summer 2019) 
\item Hieu Phan (REU, Summer 2019) 
\item Aaron Mueller (REU, Summer 2017) 
\end{itemize}

\end{rSection}

\begin{rSection}{Grants}
\begin{itemize}
\item  Social Science Research Council (SSRC)/Summer Institutes in Computational Social Science (SICSS) Research Grant 2021 (\$1200). With Ian Stewart. 
\item  Kaggle Open Data Research Grant 2020 (\$5000). With Andy Halterman and Sheikh Sarwar.
\end{itemize}

\end{rSection}

%----------------------------------------------------------------------------------------
%   INVITED TALKS 
%----------------------------------------------------------------------------------------

\begin{rSection}{Invited Talks}

\begin{itemize}
\item Williams College's Computer Science Colloquium---November 15, 2019
\item Rutgers University's Natural Language Processing Group---November 6, 2019
\item Lewis \& Clark College's Mathematical Sciences Colloquium---September 24, 2018
\end{itemize}

\end{rSection}

%----------------------------------------------------------------------------------------
%   GUEST LECTURES  
%----------------------------------------------------------------------------------------

\begin{rSection}{Guest Lectures}

\begin{itemize}
\item University of Massachusetts Amherst, Natural Language Processing---November 13, 2020 
\item Mount Holyoke College, Natural Language Processing---November 13, 2019
\item Mount Holyoke College, Natural Language Processing---October 10, 2018
\item University of Massachusetts Amherst, Advanced Natural Language Processing---April 26, 2018
\end{itemize}

\end{rSection}

%----------------------------------------------------------------------------------------
%   PRESENTATIONS AND POSTERS  
%----------------------------------------------------------------------------------------

% \begin{rSection}{Other Projects}

%  \begin{itemize}
%  \item ``Fairkit-learn:  A multi-objective optimization approach to fairness in machine learning classifiers." With Sam Witty. \emph{Advanced Software Engineering: Analysis and Evaluation} final class project (Spring 2018) 
%  \item ``Class-conditional language modeling with LSTMs." \emph{Machine Learning} final class project (Fall 2017)
%  \item ``Linguistically Motivated LSTM Architectures for Relation Extraction." With Sheshera Mysore. \emph{Neural Networks} final class project (Fall 2017)  
%  \item ``Temporal, Embedding-Based Soft Deduplication of Police Killing Events." With Su Lin Blodgett and Aaron Schein. \emph{Database Design \& Implementation} final class project (Spring 2017)  
%  \item  ``Probabilistic Modeling of Trending Words on Twitter." With Su Lin Blodgett.  \emph{Statistical Machine Learning} final class project (Fall 2016)  
%  \item ``Machine Learning Classification of Job Loss Twitter Messages." With Boya Ren. \emph{Introduction to Natural Language Processing} final class project (Fall 2016) 
%  \item ``Extending the Pontryagin Maximum Principle of Optimal Control Theory for Inequality Constraints and Discounting." \emph{Lewis \& Clark College Mathematics Department Senior Honors Thesis} (Spring 2015)
%  \item ``An Agent-Based Simulation of Intergenerational Mobility Amongst the English Medieval Peasantry." \emph{Andrew W. Mellon Student-Faculty Research Project} (Summer 2014) 
%  \end{itemize} 

% \end{rSection}

%----------------------------------------------------------------------------------------
%   SELECTED COURSES 
%----------------------------------------------------------------------------------------

% \begin{rSection}{Selected Courses}
% Machine Learning, Neural Networks, Probabilistic Graphical Models, Advanced Algorithms and Analysis, Advanced Software Engineering: Analysis and Evaluation, Database Design \& Implementation, Introduction to Natural Language Processing, Advanced Probability \& Statistics, Real Analysis, Abstract Algebra, Game Theory, Numerical Analysis, Differential Equations, Linear Algebra 
% \end{rSection}

%----------------------------------------------------------------------------------------
%   SERVICE 
%----------------------------------------------------------------------------------------

\begin{rSection}{Service \& Outreach}
 
\begin{rSubsection}{Reviewer (Program Committee)}{}{}{}
\item \emph{Science Advances}: 2021
\item \emph{Workshop of Teaching NLP}: 2021
\item \emph{ACL}: 2020, 2021
\item \emph{NAACL}: 2019, 2021
\item \emph{EACL}: 2021 
\item \emph{CoNLL}: 2020 
\item \emph{EMNLP}: 2020 
\item \emph{ICWSM}: 2019, 2020
\end{rSubsection}

\begin{rSubsection}{Service Leadership Roles}{}{}{}
\item Co-organizer, ``CS+NLP 201: Beyond the Basics'' online tutorial series (Fall 2021)
\item Co-organizer and Panel Moderator, ``First Workshop on Causal Inference and Natural Language Processing'' at EMNLP 2021

\end{rSubsection}

\begin{rSubsection}{Other Outreach Roles}{}{}{}
\item Panelist, University of Massachusetts CSWomen panel on ``Applying to Research Internships and Jobs'' (November 19, 2021)
\item Panelist, NSF CSGrad4US Mentoring Program panel on ``What I wish I knew when I started graduate school'' (November 24, 2021)
\item 
{PhD Applicant Support Mentor for Underrepresented Students} ({November 2020}) {University of Massachusetts Amherst}
\item 
{Panelist, CS Graduate School Q\&A Panel} ({May 1, 2020}) {Cape Cod Community College}

\item 
{Co-chair, CSWomen} ({February~2019--January 2020}) {University of Massachusetts Amherst Computer Science Women's Group}

\item 
{Social Co-Chair,  CSWomen} ({January 2017--January 2019}) 
{University of Massachusetts Amherst Computer Science Women's Group}

\item {Student Volunteer, Women in Engineering and Computing Career Day} ({October 28, 2019})
{University of Massachusetts Amherst}{}


\item {``What is computer science?" 7th Grade Guest Lecture} ({Jan. 3, 2018 and Jan. 7, 2019}) 
{Chief Joseph Middle School, Bozeman, Montana}

\item {Organizer, CICS Graduate Male Ally Workshop Series} ({September 2017-- May 2018}) {University of Massachusetts Amherst}

\item {Student Volunteer, Girls~Inc.~Eureka!~Summer~Workshop} ({August 1 \& 3, 2017}) {University of Massachusetts Amherst}

\item {Student Volunteer, Women in Engineering and Computing Career Day} ({October 24, 2016})
{University of Massachusetts Amherst}{}
\end{rSubsection}

\end{rSection}


%----------------------------------------------------------------------------------------
%   TECHNICAL STRENGTHS SECTION
%----------------------------------------------------------------------------------------

%    \begin{rSection}{Technical Strengths}
%    Python (scipy, scikit-learn, numpy, pandas), Pytorch, R, SQL  

%\begin{tabular}{ @{} >{\bfseries}l @{\hspace{6ex}} l }

%Machine learning modeling & CRF's, Logistic Regression, SVM's, Gibbs sampling, EM algorithm \\
%Protocols \& APIs & XML, JSON, SOAP, REST \\
%Primary programming language & Python\\
%Deep learning library & Pytorch \\
%Python modules & scipy, scikit-learn, numpy, pandas \\
%Databases & SQL, Postgres\\
%Deep learning libraries & Tensorflow, Pytorch \\
%Development Tools & sublime, emacs, tmux \\
% Data visualization: matplotlib, omnigraffle, ggplot2
%\end{tabular}

%\end{rSection}

%----------------------------------------------------------------------------------------
%   AWARDS SECTION
%----------------------------------------------------------------------------------------

\begin{rSection}{Graduate Awards \& Honors}
\begin{itemize}
\item Outstanding Reviewer, EMNLP 2020
\item Awarded \emph{distinction} for Ph.D. candidacy portfolio submitted to the College of Information and Computer Sciences at University of Massachusetts Amherst (December 2019) 
\item Bloomberg Data Science Ph.D. Fellowship (awarded May 2019)  
\item Computing Research Association of Women (CRA-W) Graduate Cohort (2017, 2018) 
\item Empirical Methods in Natural Language Processing (EMNLP) Student Travel Scholarship (2017)
\item Paul Utgoff Memorial Graduate Scholarship in Machine Learning (2016) 
\end{itemize} 
\end{rSection}

\begin{rSection}{Undergraduate Awards \& Honors}
\begin{itemize}
\item Fulbright ETA Grantee with the U.S. Department of State (2015--16)
\item Rhodes Scholarship Finalist (2015) 
\item Marshall Scholarship Finalist (2015) 
\item Rena Ratte Award, Lewis \& Clark College (2015)-- \emph{``This is the highest academic award given at Lewis \& Clark and is given annually to one undergraduate of senior standing whose abilities and commitment have combined to produce work which is consistently of the greatest distinction.''} \url{https://college.lclark.edu/dean/faculty-and-student-awards/}
% \item Robert B. Pamplin Jr. Society Fellow, Lewis \& Clark College (2012--2015)
% \item Dean's List, Lewis \& Clark College (2011--2015)
% %\item Phi Beta Kappa Member (2014--2015)
% \item Pi Mu Epsilon Member (2014--2015)
% \item Barbara Hirschi Neely Four-Year Full-Tuition Scholarship Recipient, Lewis \& Clark College (2011--2015)
\item NCAA Division III Cross-Country All-Academic (2012) 
\item Lewis \& Clark College Cross-Country Four-Year Varsity Letter Winner (2011--2015)
\end{itemize} 
\end{rSection}

%----------------------------------------------------------------------------------------
%   FOREIGN LANGUAGES 
%----------------------------------------------------------------------------------------

\begin{rSection}{Foreign Languages}
\textbf{Chinese (Mandarin)} \\
HSK Level 4 (test passed April 16, 2016) \\
CET Beijing: 16-week language-intensive immersion program (Spring 2014)
\end{rSection}

\blfootnote{Updated: \today}

\end{document}
